\documentclass{report}
\usepackage{amsmath}
\usepackage{amssymb}
\usepackage{hyperref}
\title{Formulary Linear Algebra II}


%\newcommand{\rm}{\textrm}
\newcommand{\Ker}{\textrm{Ker}}
\newcommand{\im}{\textrm{im}}
\newcommand{\vphi}{\varphi}

\renewcommand*\thesection{\arabic{section}}
\renewcommand{\chaptername}{\"Ubungs Blatt}
\renewcommand{\thesubsection}{\alph{subsection})}

\begin{document}
\chapter{}

\section{}
Trivial.
\section{}
Additive polynomial:

$$P(T_1 + T_2)=P(T_1)+P(T_2)$$
\subsection{}
Characteristic:

n summands for $1+...+1=0 \Rightarrow char(R)=n$
\subsection{}
Trivial.
\subsection{}
Trivial.
\subsection{}
Trivial.
\section{}
An integrity domain is a nonzero commutative ring in which the product of any two nonzero elements is nonzero.
\subsection{}
Injective:

$$\forall a,b \in X,f(a)=f(b) \Rightarrow a=b$$
\subsection{}
For finite fields:

injectivity $\Leftrightarrow$ surjectivity $\Leftrightarrow$ bijectivity

Requirements for a ring to be a field:

\begin{enumerate}
\item $$1\neq0$$
\item $$\forall x\in R \exists x^{-1}:xx^{-1}=x^{-1}x=1$$
\end{enumerate}
\section{}

\subsection{}
A monic polynomial is a polynomial where the leading coefficient is equal to 1.
\subsection{}
A is a divisor of B $\Rightarrow B=A\times C$.

\chapter{}
\section{•}
\subsection{•}
Equivalent matrix:
$B=T^{-1}AS$

Equivalence relation:
\begin{enumerate}
\item reflexive property $x=x$
\item symmetric property $x=y \Rightarrow y=x$
\item transitive property $a=b \land b=c \Rightarrow a=c$
\end{enumerate}
\subsection{•}
Invertible Matrix:

$$AA^{-1}=A^{-1}A=I$$

For square matrices:

$$det(A) \neq 0$$
\subsection{•}
Trivial.
\subsection{•}
Similar:
$$B=T^{-1}AS\textrm{, where }T=S$$

\section{•}
$$\dim(V)=\dim(\Ker(V))+d\im(\im(V))$$

\section{•}

\subsection{•}
Determinant:
\begin{itemize}
\item$2\times2$:

$$\det(A)=\det\begin{bmatrix}a&b\\c&d\\\end{bmatrix}=ad-bc$$
\item$3\times3$:

$$\det(A)=\det\begin{bmatrix}a&b&c\\d&e&f\\g&h&i\\\end{bmatrix}=a\det\begin{bmatrix}e&f\\h&i\\\end{bmatrix}-b\det\begin{bmatrix}d&f\\g&i\\\end{bmatrix}+c\det\begin{bmatrix}d&e\\g&h\\\end{bmatrix}$$

\item$n\times n$:

$$\det(A)=\sum_{\sigma\in S_n}(\rm{sgn}(\sigma)\prod_{i=1}^{n}a_i,\sigma_i )$$

Do note this is mostly irrelevant, since you can just perform something equivalent to a $3\times3$ determinant over and over again for larger matrices.
\end{itemize}

\subsection{}
Inverse matrix for $3\times3$ matrices:
$$A^{-1}=\frac{1}{\det(A)}\times \rm{adj}(A)$$
\url{https://en.wikipedia.org/wiki/Adjugate_matrix}

\section{}

\subsection{}
Trivial.

\subsection{}

$F(U)\subset U$ if all bases of $F(U)$ are in $U$, but $\dim(F(U))< \dim(U)$.

\chapter{}

\section{}

$$\dim(V)=\dim(U)+\dim(V/U)$$

\section{}

$$Danke$$

\subsection{}

\subsection{}

\subsection{}

\subsection{}

\section{}

A polynomial of an odd-numbered degree will always go through zero at least once.

\section{}

\subsection{}

\subsection{}

\chapter{}

\chapter{}

\chapter{}

\chapter{}

\chapter{}

\chapter{}

\chapter{}

\section{}

\label{norm}$$\|v\|=\sqrt{\sum_{i=1}^n(v_i)^2}$$

$$\rm{orthogonal }\Leftrightarrow <v,u> = 0 \Leftrightarrow \sum_{i=1}^nv_iu_i=0$$

\section{}

\url{https://en.wikipedia.org/wiki/Orthogonal_complement}
\\
\\
Norm: \ref{norm}

\section{}

\subsection{}

"$(V,<,>)$ ist ein euklidischer Raum, wenn $V$ ein endlichdimensionaler $\mathbb{R}$-Vektorraum und $<,>$ ein Skalarprodukt ist."
\\
\\
"Ein Skalarprodukt auf einem endlichdimensionalen $\mathbb{R}$-Vektorraum $V$ ist eine positiv definite symmetrische Bilinearform: $\vphi(u,v)\mapsto<u,v>$."
\\
\\
\label{posdef}A matrix M is positive definite if $\forall v>0\in \mathbb{N}: v^\top M v > 0$.
\\
\\
A symmetrical bilinear form is a bilinear form so that $B(u,v)=B(v,u)$.

\subsection{}
For $V$ vector space, $W\subset V$ a subspace, $B$ the bilinear form of $V$:
$$W^{\bot }=\left\{x\in V:B(x,y)=0{\mbox{ for all }}y\in W\right\}.$$

\subsection{}

"Eine Orthonormalbasis von $V$ [euklidischer oder unit\"arer Raum $(V,<,>)$] ist eine Basis $\mathcal{B} = \{b_i\}\textrm{ } /\textrm{ }  b_i\textrm{ }\bot\textrm{ }b_j,\textrm{ } i\neq j,\textrm{ } \|b_i\|=1$."

\subsection{}

\section{}

\subsection{}

A symmetrical matrix is a matrix $A$ so that $A=A^\top$.

\subsection{}

\ref{posdef} Positive definite matrix.

\chapter{}

\chapter{}

\end{document}