\documentclass{report}
\usepackage{amsmath}
\usepackage{esint}
\usepackage{hyperref}
\usepackage[utf8]{inputenc}

%\newcommand{\rm}{\textrm}
\newcommand{\vphi}{\varphi}
\newcommand{\lag}{\mathcal{L}}
\newcommand{\ham}{\mathcal{H}}
\newcommand{\prt}[2]{\frac{\partial #1}{\partial #2}}

\title{Formulary Theoretical Physics I}

\renewcommand*\thesection{\arabic{section}}
\renewcommand{\chaptername}{Übungs Blatt}
\renewcommand{\thesubsection}{\alph{subsection})}

\begin{document}
\chapter{}

\section{}

\subsection{}
$$m\vec{\ddot{r}}=-m\vec{g}-2m(\vec{\omega}\times\vec{\dot{r}}^{'})+m\vec{\omega}\times(\vec{\omega}\times\vec{r}^{'})$$

For the earth: $\omega^2 \ll \omega$
\subsection{}
Probably trivial, but:
$$\vec{\omega}^{'}=\omega (-\cos(\phi_0),0,\sin(\phi_0))$$
\subsection{}
Trivial.
\subsection{}
Trivial.
\section{}

\subsection{}
Trivial.
\subsection{}
Trivial.
\section{}

\subsection{}
Trivial (differential equations).
\subsection{}
Trivial.
\chapter{}

\section{}

\subsection{}
$V_s=\frac{\sum_i m_i v_i}{\sum_i m_i}$

\subsection{}

\subsection{}

\section{}
Gauss's law for gravity:

$$\oiint_{\displaystyle \scriptstyle \partial V} {\displaystyle \mathbf {g} \cdot d\mathbf {A} =-4\pi GM} \mathbf {g} \cdot d\mathbf {A} =-4\pi GM$$

\section{}

\subsection{}

\subsection{}

\chapter{}

\chapter{}

\chapter{}

\chapter{}

\chapter{}

\chapter{}

\chapter{}

\chapter{}

\section{}

\subsection{}

$$V=\rho\int_{-a/2}^{a/2}\int_{-b/2}^{b/2}\int_{-c/2}^{c/2}[(\vec{r}^2\delta_{ij}-r_ir_j]dV$$

\subsection{}

$$\frac{1}{2}\omega^T(V)\omega$$

\section{}

\subsection{}

$$p_\phi=I_1\sin^2\theta\dot\vphi+I_3(\dot\psi+\dot\vphi\cos\theta)\cos\theta$$
$$p_\psi=I_3(\dot\psi+\dot\vphi\cos\theta)$$

\subsection{}
Einsetzen.

\subsection{}
$$\rm{Taylor Series: }{f(a)+{\frac {f'(a)}{1!}}(x-a)+{\frac{f''(a)}{2!}}(x-a)^{2}+{\frac{f'''(a)}{3!}}(x-a)^{3}+\cdots}$$

$$\rm{L'Hopital: }\lim_{x\rightarrow c}\frac{f(x)}{g(x)}= \lim_{x\rightarrow c}\frac{f'(x)}{g'(x)}$$


\section{}
\begin{description}
\item [Hamilton Funktion:]
$$\ham(q,p,t):={\sum_{i=1}\dot q_i p_i}- \lag( q , \dot q , t ) ,  \textrm{ with  }\dot q = \dot q ( q , p , t )$$
\item [Hamiltonischen Gleichungen:]
$$\dot q_i = \frac{\partial \ham}{\partial p_i} \qquad \dot p_i = -\frac{\partial\ham}{\partial q_i} \qquad i = 1, \dots n$$
\item [Totale Zeit Ableitung]
$$\frac{d \ham}{dt} = \sum_{i=1}^{n} \bigg ( \prt{\ham}{q_i} \dot q_i + \prt{\ham}{p_i} \dot p_i \bigg ) + \prt{\ham}{t}$$
\end{description}
\section{}

kek

\chapter{Assignment Sheet 11}



\end{document}